\documentclass{ieeeaccess}
\pdfmapfile{+t1-formata.map}
\pdfmapfile{+t1-giovannistd.map}
\pdfmapfile{+t1-helvetica.map}
\pdfmapfile{+t1-times.map}
\usepackage{cite}
\usepackage{amsmath,amssymb,amsfonts}
\usepackage{algorithmic}
\usepackage{graphicx}
\usepackage{textcomp}
\graphicspath{{figures/}}
\usepackage{kotex}

\usepackage{bm}
\makeatletter
\AtBeginDocument{\DeclareMathVersion{bold}
\SetSymbolFont{operators}{bold}{T1}{times}{b}{n}
\SetSymbolFont{NewLetters}{bold}{T1}{times}{b}{it}
\SetMathAlphabet{\mathrm}{bold}{T1}{times}{b}{n}
\SetMathAlphabet{\mathit}{bold}{T1}{times}{b}{it}
\SetMathAlphabet{\mathbf}{bold}{T1}{times}{b}{n}
\SetMathAlphabet{\mathtt}{bold}{OT1}{pcr}{b}{n}
\SetSymbolFont{symbols}{bold}{OMS}{cmsy}{b}{n}
\renewcommand\boldmath{\@nomath\boldmath\mathversion{bold}}}
\makeatother

\def\BibTeX{{\rm B\kern-.05em{\sc i\kern-.025em b}\kern-.08em
    T\kern-.1667em\lower.7ex\hbox{E}\kern-.125emX}}

%Your document starts from here ___________________________________________________
\begin{document}
\history{Date of publication xxxx 00, 0000, date of current version xxxx 00, 0000.}
\doi{10.1109/ACCESS.2024.0429000}

\title{IEEE ACCESS 논문 투고를 위한 준비}
\author{\uppercase{First A. Author}\authorrefmark{1}, \IEEEmembership{Fellow, IEEE},
\uppercase{Second B. Author}\authorrefmark{2}, and Third C. Author,
Jr.\authorrefmark{3},
\IEEEmembership{Member, IEEE}}

\address[1]{National Institute of Standards and
Technology, Boulder, CO 80305 USA (e-mail: author@boulder.nist.gov)}
\address[2]{Department of Physics, Colorado State University, Fort Collins,
CO 80523 USA (e-mail: author@lamar.colostate.edu)}
\address[3]{Electrical Engineering Department, University of Colorado, Boulder, CO
80309 USA}
\tfootnote{첫 번째 각주의 이 단락에는 스폰서 및 재정 지원에 대한 감사를 포함한 지원 정보가 포함됩니다. 예를 들어, ``This work was supported in part by the U.S. Department of Commerce under Grant BS123456.''와 같이 작성합니다.}

\markboth
{저자 \headeretal: IEEE TRANSACTIONS 및 JOURNALS 논문 준비}
{저자 \headeretal: IEEE TRANSACTIONS 및 JOURNALS 논문 준비}

\corresp{Corresponding author: First A. Author (e-mail: author@ boulder.nist.gov).}


\begin{abstract}
이 지침은 IEEE Access 논문을 준비하기 위한 가이드라인을 제공합니다. \LaTeX을 사용하는 경우 이 문서를 템플릿으로 사용하십시오. 그렇지 않은 경우, 이 문서를 지침서로 활용하십시오. 논문의 전자 파일은 IEEE에서 추가로 서식이 지정됩니다. 논문 제목은 모두 대문자가 아닌, 대문자와 소문자를 혼용하여 작성해야 합니다. 제목에 아래 첨자가 포함된 긴 수식은 피하고, 요소를 식별하는 짧은 수식은 괜찮습니다(예: "Nd--Fe--B"). 제목에 "(Invited)"라고 쓰지 마십시오. 저자란에는 저자의 성명을 전체 이름으로 기재하는 것이 좋지만 필수는 아닙니다. 저자의 이니셜 사이에는 공백을 두십시오. 초록(Abstract)은 논문의 내용을 간결하면서도 포괄적으로 반영해야 합니다. 특히 초록은 약어, 각주, 참고문헌 없이 그 자체로 완결성을 가져야 합니다. 이는 전체 논문의 축소판이어야 합니다. 초록은 150~250단어 사이여야 합니다. 이 제한을 반드시 준수하십시오. 그렇지 않으면 초록을 수정해야 할 수도 있습니다. 초록은 하나의 문단으로 작성해야 하며, 수식이나 표를 포함해서는 안 됩니다. 초록에는 독자가 논문을 쉽게 찾을 수 있도록 3~4개의 서로 다른 키워드나 문구를 포함해야 합니다. 검색 엔진에서 페이지가 거부될 수 있으므로 이러한 문구가 과도하게 반복되지 않도록 주의하는 것이 중요합니다. 초록이 문법적으로 정확하고 잘 읽히는지 확인하십시오.
\end{abstract}

\begin{keywords}
알파벳 순서(가나다 순)로 키워드나 문구를 쉼표로 구분하여 입력하십시오. 자기상관, 빔포밍, 통신 기술, 사전 학습, 피드백, fMRI, mmWave, 다중 경로, 시스템 설계, 미세 결함, 윤활 부족 결함.
\end{keywords}

\titlepgskip=-21pt

\maketitle

\section{Introduction}
\label{sec:introduction}
\PARstart{T}{his} section introduces the background and motivation of the research. Recent advancements in artificial intelligence have revolutionized various fields. However, there remain significant challenges in optimizing these systems for real-world applications.

This paper proposes a novel approach to address these limitations. We begin by discussing the current state of the art and identifying the gaps that our research aims to fill. The main contributions of this paper are summarized as follows:
\begin{itemize}
    \item We propose a new framework for efficient data processing.
    \item We demonstrate the effectiveness of our method through extensive experiments.
    \item We provide a comprehensive analysis of the results compared to existing methods.
\end{itemize}

The remainder of this paper is organized as follows. Section \ref{sec:related_work} reviews related work. Section \ref{sec:methodology} details our proposed methodology. Section \ref{sec:experiments} presents the experimental setup and results. Finally, Section \ref{sec:conclusion} concludes the paper.

\section{Units}
SI (MKS) 또는 CGS를 기본 단위로 사용하십시오. (SI 단위 사용을 강력히 권장합니다.) 영국식 단위는 보조 단위로 괄호 안에 사용할 수 있습니다. 이는 데이터 저장 관련 논문에 적용됩니다. 예를 들어, ``15 Gb/cm$^{2}$ (100 Gb/in$^{2})$''라고 씁니다. 예외적으로 ``3$^{1\!/\!2}$-in 디스크 드라이브''와 같이 무역에서 식별자로 사용되는 경우 영국식 단위를 사용합니다. 전류는 암페어, 자기장은 에르스테드와 같이 SI와 CGS 단위를 혼용하지 마십시오. 이는 차원적으로 균형이 맞지 않아 혼란을 초래하는 경우가 많습니다. 혼합 단위를 사용해야 하는 경우, 방정식에서 각 수량에 대한 단위를 명확하게 명시하십시오.

자기장 강도 $H$의 SI 단위는 A/m입니다. 그러나 T 단위를 사용하고 싶다면, 자속 밀도 $B$를 참조하거나 $\mu _{0}H$로 기호화된 자기장 강도를 참조하십시오. 복합 단위는 가운데 점을 사용하여 구분하십시오(예: ``A$\cdot $m$^{2}$'').


\Figure[t!](topskip=0pt, botskip=0pt, midskip=0pt){fig1.png}
{ \mbox{인가된 자기장의 함수로서의 자화. 캡션에서 그림의 중요성을 설명하는 것이 좋습니다.}\label{fig1}}

\section{Some Common Mistakes}
``data''라는 단어는 단수가 아니라 복수입니다. 진공 투자율 $\mu _{0}$의 아래 첨자는 소문자 ``o''가 아니라 숫자 0입니다. 잔류 자화에 대한 용어는 ``remanence''이며, 형용사는 ``remanent''입니다. ``remnance'' 또는 ``remnant''라고 쓰지 마십시오. ``micron'' 대신 ``micrometer''라는 단어를 사용하십시오. 그래프 내의 그래프는 ``insert''가 아니라 ``inset''입니다. ``alternatively''라는 단어는 ``alternately''(교대로 일어나는 것을 의미하지 않는 한)보다 선호됩니다. ``while''(동시 사건을 언급하지 않는 한) 대신 ``whereas''를 사용하십시오. ``approximately'' 또는 ``effectively''를 의미하기 위해 ``essentially''라는 단어를 사용하지 마십시오. ``problem''의 완곡한 표현으로 ``issue''라는 단어를 사용하지 마십시오. 조성이 지정되지 않은 경우 화학 기호를 엔 대시로 구분하십시오. 예를 들어, ``NiMn''은 금속 간 화합물 Ni$_{0.5}$Mn$_{0.5}$를 나타내는 반면, ``Ni--Mn''은 Ni$_{x}$Mn$_{1-x}$ 조성의 합금을 나타냅니다.

동음이의어인 ``affect''(주로 동사)와 ``effect''(주로 명사), ``complement''와 ``compliment'', ``discreet''와 ``discrete'', ``principal''(예: ``principal investigator'')과 ``principle''(예: ``principle of measurement'')의 의미 차이를 유의하십시오. ``imply''와 ``infer''를 혼동하지 마십시오.

``non'', ``sub'', ``micro'', ``multi'', ``ultra''와 같은 접두사는 독립적인 단어가 아닙니다. 일반적으로 하이픈 없이 수식하는 단어에 붙여 써야 합니다. 라틴어 약어 ``\emph{et al.}''의 ``et'' 뒤에는 마침표가 없습니다(이탤릭체로 표기). 약어 ``i.e.,''는 ``즉''을 의미하고, ``e.g.,''는 ``예를 들어''를 의미합니다(이 약어들은 이탤릭체로 표기하지 않습니다).

일반적인 IEEE 스타일 가이드는 다음에서 확인할 수 있습니다.\break
\underline{http://www.ieee.org/authortools}.

\section{Guidelines for Graphics Preparation and Submission}
\label{sec:guidelines}

\subsection{Types of Graphics}
다음 목록은 IEEE 저널에 게재되는 다양한 유형의 그래픽을 요약한 것입니다. 구성 및 색상/회색 음영 사용에 따라 분류됩니다.

\subsubsection{Color/Grayscale figures}
{컬러 또는 검정/회색 음영으로 나타나도록 의도된 그림입니다. 이러한 그림에는 사진, 일러스트레이션, 다색 그래프 및 순서도가 포함될 수 있습니다. 다색 그래프의 경우 회색 배경이나 음영, 스크린샷은 피하고 대신 데이터를 수집하는 데 사용된 프로그램에서 그래프를 내보내십시오.}

\subsubsection{Line Art figures}
{검은색 선과 도형으로만 구성된 그림입니다. 이러한 그림에는 회색 음영이나 하프톤이 없어야 하며 검은색과 흰색만 있어야 합니다.}

\subsubsection{Author photos}
{저자 사진은 참고문헌 아래 기사 끝에 위치한 저자 약력과 함께 포함되어야 합니다.}

\subsubsection{Tables}
{일반적으로 흑백이지만 때로는 색상이 포함된 데이터 차트입니다.}

\begin{table}
\caption{\textbf{자기 특성 단위}}
\label{table}
\setlength{\tabcolsep}{3pt}
\begin{tabular}{|p{25pt}|p{75pt}|p{115pt}|}
\hline
기호&
수량&
가우스 및 \par CGS EMU에서 SI로의 변환 $^{\mathrm{a}}$ \\
\hline
$\Phi $&
자속&
1 Mx $\to  10^{-8}$ Wb $= 10^{-8}$ V$\cdot $s \\
$B$&
자속 밀도, \par 자기 유도&
1 G $\to  10^{-4}$ T $= 10^{-4}$ Wb/m$^{2}$ \\
$H$&
자기장 강도&
1 Oe $\to  10^{3}/(4\pi )$ A/m \\
$m$&
자기 모멘트&
1 erg/G $=$ 1 emu \par $\to 10^{-3}$ A$\cdot $m$^{2} = 10^{-3}$ J/T \\
$M$&
자화&
1 erg/(G$\cdot $cm$^{3}) =$ 1 emu/cm$^{3}$ \par $\to 10^{3}$ A/m \\
4$\pi M$&
자화&
1 G $\to  10^{3}/(4\pi )$ A/m \\
$\sigma $&
비자화&
1 erg/(G$\cdot $g) $=$ 1 emu/g $\to $ 1 A$\cdot $m$^{2}$/kg \\
$j$&
자기 쌍극자 \par 모멘트&
1 erg/G $=$ 1 emu \par $\to 4\pi \times  10^{-10}$ Wb$\cdot $m \\
$J$&
자기 분극&
1 erg/(G$\cdot $cm$^{3}) =$ 1 emu/cm$^{3}$ \par $\to 4\pi \times  10^{-4}$ T \\
$\chi , \kappa $&
자화율&
1 $\to  4\pi $ \\
$\chi_{\rho }$&
질량 자화율&
1 cm$^{3}$/g $\to  4\pi \times  10^{-3}$ m$^{3}$/kg \\
$\mu $&
투자율&
1 $\to  4\pi \times  10^{-7}$ H/m \par $= 4\pi \times  10^{-7}$ Wb/(A$\cdot $m) \\
$\mu_{r}$&
상대 투자율&
322: $\mu \to \mu_{r}$ \\
$w, W$&
에너지 밀도&
1 erg/cm$^{3} \to  10^{-1}$ J/m$^{3}$ \\
$N, D$&
반자화 계수&
1 $\to  1/(4\pi )$ \\
\hline
\multicolumn{3}{p{251pt}}{표의 세로선은 선택 사항입니다. 전체 표의 캡션 역할을 하는 문장에는 각주 문자가 필요하지 않습니다.}\\
\multicolumn{3}{p{251pt}}{$^{\mathrm{a}}$가우스 단위는 정자기에 대한 cgs emu와 동일합니다. Mx
$=$ 맥스웰, G $=$ 가우스, Oe $=$ 에르스테드; Wb $=$ 웨버, V $=$ 볼트, s $=$
초, T $=$ 테슬라, m $=$ 미터, A $=$ 암페어, J $=$ 줄, kg $=$
킬로그램, H $=$ 헨리.}
\end{tabular}
\label{tab1}
\end{table}

\subsection{Multipart figures}
둘 이상의 하위 그림이 나란히 또는 쌓여서 구성된 그림입니다. 멀티파트 그림이 여러 그림 유형(한 부분은 라인 아트이고 다른 부분은 회색 음영 또는 컬러)으로 구성된 경우 그림은 더 엄격한 지침을 충족해야 합니다.

\subsection{File Formats For Graphics}\label{formats}
적절한 그래픽 처리 프로그램을 사용하여 그래픽의 형식을 지정하고 저장하십시오. 이 프로그램을 사용하면 이미지를 PostScript(.PS), Encapsulated PostScript(.EPS), Tagged Image File Format(.TIFF), Portable Document Format(.PDF), Portable Network Graphics(.PNG) 또는 Metapost(.MPS)로 생성하고 크기를 조정하고 해상도 설정을 조정할 수 있습니다. 최종 논문을 제출할 때 모든 그래픽은 원고와 함께 이러한 형식 중 하나로 개별적으로 제출해야 합니다.

\subsection{Sizing of Graphics}
대부분의 차트, 그래프 및 표는 1열 너비(3.5인치/88밀리미터/21파이카) 또는 페이지 너비(7.16인치/181밀리미터/43파이카)입니다. 그래픽의 최대 깊이는 8.5인치(216밀리미터/54파이카)일 수 있습니다. 그래픽의 깊이를 선택할 때는 캡션을 위한 공간을 확보하십시오. 저자가 선택하는 경우 그림 크기를 열 너비와 페이지 너비 사이로 조정할 수 있지만, 필요한 경우가 아니면 그림 크기를 열 너비보다 작게 조정하지 않는 것이 좋습니다.

현재 위에 나열된 것과 일치하지 않는 열 치수를 가진 간행물이 하나 있습니다. Proceedings of the IEEE의 열 치수는 3.25인치(82.5밀리미터/19.5파이카)입니다.

저자 사진의 최종 인쇄 크기는 정확히 가로 1인치, 세로 1.25인치(25.4밀리미터$\,\times\,$31.75밀리미터/6파이카$\,\times\,$7.5파이카)입니다. 사설에 인쇄된 저자 사진은 가로 1.59인치, 세로 2인치(40밀리미터$\,\times\,$50밀리미터/9.5파이카$\,\times\,$12파이카)입니다.

\subsection{Resolution }
그림의 적절한 해상도는 ``그림 유형'' 섹션에 정의된 대로 그림 유형에 따라 달라집니다. 저자 사진, 컬러 및 회색 음영 그림은 최소 300dpi여야 합니다. 표를 포함한 라인 아트는 최소 600dpi여야 합니다.

\subsection{Vector Art}
여러 컴퓨터 플랫폼에서 그림의 무결성을 보존하기 위해 .EPS/.PDF/.PS 형식의 파일을 허용합니다. 최상의 품질 결과를 얻으려면 모든 글꼴을 포함하거나 텍스트를 윤곽선으로 변환해야 합니다.

\subsection{Color Space}
색 공간이라는 용어는 해당 매체 내에서 표현할 수 있는 색상의 전체 합계를 나타냅니다. 우리의 목적을 위해 세 가지 주요 색 공간은 회색 음영, RGB(빨강/초록/파랑) 및 CMYK(시안/마젠타/노랑/검정)입니다. RGB는 일반적으로 화면 그래픽에 사용되는 반면 CMYK는 인쇄 목적으로 사용됩니다.

모든 컬러 그림은 RGB 또는 CMYK 색 공간에서 생성되어야 합니다. 회색 음영 이미지는 회색 음영 색 공간으로 제출해야 합니다. 라인 아트는 회색 음영 또는 비트맵 색 공간으로 제공될 수 있습니다. ``비트맵 색 공간''과 ``비트맵 파일 형식''은 같은 것이 아닙니다. 비트맵 색 공간을 선택한 경우 .TIF/.TIFF/.PNG가 권장되는 파일 형식입니다.

\subsection{Accepted Fonts Within Figures}
그래픽을 준비할 때 IEEE는 다음 오픈 타입 글꼴 중 하나를 사용할 것을 제안합니다: Times New Roman, Helvetica, Arial, Cambria 및 Symbol. EPS, PS 또는 PDF 파일을 제공하는 경우 모든 글꼴이 포함되어야 합니다. 일부 글꼴은 운영 체제에만 고유할 수 있습니다. 글꼴이 포함되지 않으면 그래픽의 일부가 왜곡되거나 누락될 수 있습니다.

그림을 완성할 때 안전한 옵션은 파일을 저장하기 전에 글꼴을 제거하여 ``윤곽선'' 유형을 만드는 것입니다. 이렇게 하면 글꼴이 아트워크로 변환되어 모든 화면에서 균일하게 나타납니다.

\subsection{Using Labels Within Figures}

\subsubsection{Figure Axis labels }
그림 축 라벨은 종종 혼란의 원인이 됩니다. 기호보다는 단어를 사용하십시오. 예를 들어, 단순히 ``M''이 아니라 ``Magnetization'' 또는 ``Magnetization M''이라고 쓰십시오. 단위는 괄호 안에 넣으십시오. 축에 단위만 라벨로 붙이지 마십시오. 예를 들어 그림 1과 같이 단순히 ``A/m''이 아니라 ``Magnetization (A/m)'' 또는 ``Magnetization (A$\cdot$m$^{-1}$)''이라고 쓰십시오. 축에 수량과 단위의 비율로 라벨을 붙이지 마십시오. 예를 들어 ``Temperature/K''가 아니라 ``Temperature (K)''라고 쓰십시오.

승수는 특히 혼란스러울 수 있습니다. ``Magnetization (kA/m)'' 또는 ``Magnetization (10$^{3}$ A/m)''라고 쓰십시오. ``Magnetization (A/m)$\,\times\,$1000''이라고 쓰지 마십시오. 독자가 그림 1의 상단 축 라벨이 16000 A/m를 의미하는지 0.016 A/m를 의미하는지 알 수 없기 때문입니다. 그림 라벨은 읽기 쉬워야 하며 약 8~10포인트 크기여야 합니다.

\subsubsection{Subfigure Labels in Multipart Figures and Tables}
멀티파트 그림은 최종 제출 전에 결합하고 라벨을 붙여야 합니다. 라벨은 (a) (b) (c) 형식의 8포인트 Times New Roman 글꼴로 각 하위 그림 아래 중앙에 표시되어야 합니다.

\subsection{File Naming}
그림(라인 아트워크 또는 사진)은 저자의 성의 처음 5글자로 시작하여 이름을 지정해야 합니다. 파일 이름의 다음 문자는 기사에서 이 이미지의 순차적 위치를 나타내는 숫자여야 합니다. 예를 들어, 저자 ``Anderson''의 논문에서 처음 세 그림의 이름은 ander1.tif, ander2.tif, ander3.ps가 됩니다.

표는 표의 본문(캡션 아님)만 포함해야 하며, 저자의 이름과 표 번호 사이에 `.t`가 삽입된다는 점을 제외하고는 그림과 유사하게 이름을 지정해야 합니다. 예를 들어, 저자 Anderson의 처음 세 표의 이름은 ander.t1.tif, ander.t2.ps, ander.t3.eps가 됩니다.

저자 사진은 사진에 찍힌 저자의 성의 처음 5글자를 사용하여 이름을 지정해야 합니다. 예를 들어, 논문의 저자 사진 4장은 oppen.ps, moshc.tif, chen.eps, duran.pdf로 명명될 수 있습니다.

두 명 이상의 저자가 같은 성을 가진 경우, 구별이 될 때까지 성의 다섯 번째, 네 번째, 세 번째$\ldots$ 문자 대신 첫 번째 이니셜을 대체할 수 있습니다. 예를 들어, 두 저자 Michael과 Monica Oppenheimer의 사진은 oppmi.tif와 oppmo.eps로 명명됩니다.

\subsection{Referencing a Figure or Table Within Your Paper}
논문 내에서 그림과 표를 참조할 때는 문장의 시작 부분이라도 ``Fig.''라는 약어를 사용하십시오. 그림은 아라비아 숫자로 번호를 매겨야 합니다. ``Table''을 줄여 쓰지 마십시오. 표는 로마 숫자로 번호를 매겨야 합니다.

\subsection{Submitting Your Graphics}
그림은 섹션 IV-C에 나열된 파일 형식 중 하나로 원고와 별도로 개별적으로 제출해야 합니다. 그림 캡션은 그림 아래에 배치하고 표 제목은 표 위에 배치하십시오. 캡션을 그림의 일부로 포함하거나 그림에 연결된 ``텍스트 상자''에 넣지 마십시오. 또한 그림 외부에 테두리를 두지 마십시오.

\subsection{Color Processing/Printing in IEEE Journals}
모든 IEEE Transactions, Journals 및 Letters는 저자가 IEEE {\it Xplore}$\circledR$에 컬러 그림을 무료로 게시할 수 있도록 허용하며, 인쇄 버전의 경우 자동으로 회색 음영으로 변환합니다. 대부분의 저널에서 저자가 선택하는 경우 그림과 표를 컬러로 인쇄할 수도 있습니다. 이 서비스에는 저자에게 추가 비용이 발생한다는 점에 유의하십시오. 컬러 그래픽 인쇄를 원하는 경우, 최종 논문에 어떤 그림이나 표를 그렇게 처리할지 명시하고 추가 비용을 지불할 의사가 있음을 명시한 메모를 포함하십시오.

\section{Conclusion}
결론에서 논문의 주요 요점을 검토할 수는 있지만, 초록을 결론으로 복제하지는 마십시오. 결론에서는 연구의 중요성을 자세히 설명하거나 응용 및 확장 가능성을 제안할 수 있습니다.

여러 개의 부록이 있는 경우 아래의 $\backslash$appendices 명령을 사용하십시오. 부록이 하나만 있는 경우 $\backslash$appendix[부록 제목]을 사용하십시오.

\appendices
\section{\break Footnotes}
각주 번호는 위첨자 숫자로 별도로 매깁니다.\footnote{각주는 피하는 것이 좋습니다(첫 페이지의 접수 날짜가 있는 번호 없는 각주 제외). 대신 각주 정보를 본문에 통합하도록 노력하십시오.} 실제 각주는 인용된 열의 맨 아래에 배치하십시오. 참고문헌 목록(미주)에 각주를 넣지 마십시오. 표 각주에는 문자를 사용하십시오(Table \ref{table} 참조).

\section{\break Submitting Your Paper for Review}

\subsection{Final Stage}
논문이 게재 승인되면, 논문을 제출한 시스템을 통해 저널의 지침에 따라 그림, 표, 사진을 포함한 최종 파일을 제출할 수 있습니다. 대용량 파일의 경우 \emph{Zip}을 사용하거나 \emph{Compress, Pkzip, Stuffit,} 또는 \emph{Gzip}을 사용하여 파일을 압축할 수 있습니다.

또한 저자 중 한 명을 ``교신 저자''로 지정하십시오. 이 저자에게 논문의 교정본이 발송됩니다. 교정본은 교신 저자에게만 발송됩니다.

\subsection{Review Stage Using IEEE Author Portal}
IEEE Access에 대한 논문 기고는 IEEE Author Portal에서 전자적으로 제출해야 합니다. 자세한 내용은 \underline{https://ieeeaccess.ieee.org/}를 방문하십시오.

다른 정보와 함께 드롭다운 목록에서 주제를 선택하라는 요청을 받게 됩니다. 제출 과정에는 여러 단계가 있으며, 완전한 제출을 위해서는 모든 단계를 완료해야 합니다. 각 단계가 끝나면 ``Save and Continue''를 클릭해야 합니다. 단순히 논문을 업로드하는 것만으로는 충분하지 않습니다. 마지막 단계가 끝나면 제출이 완료되었다는 확인 메시지가 표시되어야 합니다. 또한 이메일 확인도 받아야 합니다. 논문 제출에 관한 문의 사항은 ieeeaccess@ieee.org로 연락하십시오.

원고는 필수 IEEE Access 템플릿을 사용하여 2단, 단일 간격 형식으로 준비해야 합니다. IEEE Author Portal에 제출할 때는 Word 또는 LaTeX 파일과 PDF 파일이 모두 필요합니다.

\subsection{Final Stage Using IEEE Author Portal}
게재가 승인되면 구체적인 지침이 담긴 이메일을 받게 됩니다.

IEEE Author Portal에서 원고를 제출한 저자를 ``교신 저자''로 지정하십시오. 이 저자는 논문의 교정본을 받을 유일한 저자입니다.

\subsection{Copyright Form}
저자는 최종 원고 파일을 제출할 때 전자 IEEE 저작권 양식(eCF)을 제출해야 합니다. 원고 제출 시스템이나 Author Gateway를 통해 eCF 시스템에 액세스할 수 있습니다. 필요한 승인 및/또는 보안 허가를 받는 것은 귀하의 책임입니다. 지적 재산권에 대한 자세한 내용은 IEEE 지적 재산권 부서 웹 페이지(\underline{http://www.ieee.org/publications\_standards/publications/}\break\underline{rights/index.html})를 방문하십시오.

\section{\break IEEE Publishing Policy}
일반적인 IEEE 정책에 따르면 저자는 다른 곳에 출판된 적이 없으며 다른 심사 저널에서 검토 중이지 않은 독창적인 저작물만 제출해야 합니다. 제출 저자는 원고를 제출할 때 모든 이전 출판물과 현재 제출물을 공개해야 합니다. ``예비'' 데이터나 결과를 게시하지 마십시오. 출판 지연을 방지하려면 이 지침을 반드시 따르십시오. 최종 제출물에는 승인된 원고의 소스 파일, 고품질 그래픽 파일 및 서식이 지정된 pdf 파일이 포함되어야 합니다. 최종 제출 과정에 대해 질문이 있는 경우 저널의 관리 담당자에게 문의하십시오. 저자는 논문을 제출하기 전에 모든 공동 저자의 동의와 고용주 또는 스폰서로부터 필요한 동의를 얻을 책임이 있습니다.

IEEE Access 편집국은 컨퍼런스 기록이나 프로시딩을 출판하지 않지만, 엄격한 동료 심사를 거친 컨퍼런스 관련 기사는 출판할 수 있습니다. 동료 심사를 위해 제출된 모든 기사에는 최소 두 번의 심사가 필요합니다.

\section{\break Publication Principles}
저자는 다음 사항을 고려해야 합니다.
\begin{enumerate}
\item 출판을 위해 제출된 기술 논문은 지식의 상태를 발전시켜야 하며 관련 선행 연구를 인용해야 합니다.
\item 제출된 논문의 길이는 연구의 중요성에 비례하거나 복잡성에 적절해야 합니다. 예를 들어, 이전에 출판된 연구의 명백한 확장은 출판에 적합하지 않거나 몇 페이지만으로 충분히 다룰 수 있을지 모릅니다.
\item 저자는 동료 심사 위원과 편집자 모두에게 논문의 과학적 및 기술적 가치를 설득해야 합니다. 놀랍거나 예상치 못한 결과가 보고될 때는 입증 기준이 더 높습니다.
\item 과학적 진보를 위해서는 재현이 필요하므로, 출판을 위해 제출된 논문은 독자가 유사한 실험이나 계산을 수행하고 보고된 결과를 사용할 수 있도록 충분한 정보를 제공해야 합니다. 모든 것을 공개할 필요는 없지만, 논문에는 새롭고 유용하며 완전히 설명된 정보가 포함되어야 합니다. 예를 들어, 논문의 주된 목적이 새로운 측정 기술을 소개하는 것이라면 시료의 화학적 조성을 보고할 필요는 없습니다. 결과가 적절한 데이터와 중요한 세부 사항으로 뒷받침되지 않는 경우 저자는 심사 위원으로부터 이의 제기를 받을 것으로 예상해야 합니다.
\item 진행 중인 작업을 설명하거나 최신 기술 성과를 발표하는 논문은 전문 컨퍼런스 발표에는 적합할 수 있지만 출판에는 적합하지 않을 수 있습니다.
\end{enumerate}

\raggedbottom
\section{Reference Examples}

\begin{itemize}

\item {\bfseries 책의 기본 형식:}\\
J. K. Author, ``Title of chapter in the book,'' in \emph{Title of His Published Book, x}th ed. City of Publisher, (only U.S. State), Country: Abbrev. of Publisher, year, ch. $x$, sec. $x$, pp. \emph{xxx--xxx.}\\
참조 \cite{b1,b2}.

\item {\bfseries 정기간행물의 기본 형식:}\\
J. K. Author, ``Name of paper,'' \emph{Abbrev. Title of Periodical}, vol. \emph{x, no}. $x, $pp\emph{. xxx--xxx, }Abbrev. Month, year, DOI. 10.1109.\emph{XXX}.123456.\\
참조 \cite{b3}--\cite{b5}.

\item {\bfseries 보고서의 기본 형식:}\\
J. K. Author, ``Title of report,'' Abbrev. Name of Co., City of Co., Abbrev. State, Country, Rep. \emph{xxx}, year.\\
참조 \cite{b6,b7}.

\item {\bfseries 핸드북의 기본 형식:}\\
\emph{Name of Manual/Handbook, x} ed., Abbrev. Name of Co., City of Co., Abbrev. State, Country, year, pp. \emph{xxx--xxx.}\\
참조 \cite{b8,b9}.

\item {\bfseries 책의 기본 형식(온라인 이용 가능 시):}\\
J. K. Author, ``Title of chapter in the book,'' in \emph{Title of
Published Book}, $x$th ed. City of Publisher, State, Country: Abbrev.
of Publisher, year, ch. $x$, sec. $x$, pp. \emph{xxx--xxx}. [Online].
Available: \url{http://www.web.com}\\
참조 \cite{b10}--\cite{b13}.

\item {\bfseries 저널의 기본 형식(온라인 이용 가능 시):}\\
J. K. Author, ``Name of paper,'' \emph{Abbrev. Title of Periodical}, vol. $x$, no. $x$, pp. \emph{xxx--xxx}, Abbrev. Month, year. Accessed on: Month, Day, year, DOI: 10.1109.\emph{XXX}.123456, [Online].\\
참조 \cite{b14}--\cite{b16}.

\item {\bfseries 컨퍼런스 발표 논문의 기본 형식(온라인 이용 가능 시): }\\
J.K. Author. (year, month). Title. presented at abbrev. conference title. [Type of Medium]. Available: site/path/file\\
참조 \cite{b17}.

\item {\bfseries 보고서 및 핸드북의 기본 형식(온라인 이용 가능 시):}\\
J. K. Author. ``Title of report,'' Company. City, State, Country. Rep. no., (optional: vol./issue), Date. [Online] Available: site/path/file\\
참조 \cite{b18,b19}.

\item {\bfseries 컴퓨터 프로그램 및 전자 문서의 기본 형식(온라인 이용 가능 시): }\\
Legislative body. Number of Congress, Session. (year, month day). \emph{Number of bill or resolution}, \emph{Title}. [Type of medium]. Available: site/path/file\\
참조 \cite{b20}.

\item {\bfseries 특허의 기본 형식(온라인 이용 가능 시):}\\
Name of the invention, by inventor's name. (year, month day). Patent Number [Type of medium]. Available: site/path/file\\
참조 \cite{b21}.

\item {\bfseries 컨퍼런스 프로시딩의 기본 형식(출판됨):}\\
J. K. Author, ``Title of paper,'' in \emph{Abbreviated Name of Conf.}, City of Conf., Abbrev. State (if given), Country, year, pp. \emph{xxxxxx.}\\
참조 \cite{b22}.

\item {\bfseries 컨퍼런스 발표 논문의 예(미출판):}\\
참조 \cite{b23}.

\item {\bfseries 특허의 기본 형식}$:$\\
J. K. Author, ``Title of patent,'' U.S. Patent \emph{x xxx xxx}, Abbrev. Month, day, year.\\
참조 \cite{b24}.

\item {\bfseries 학위 논문(석사) 및 박사 학위 논문의 기본 형식:}
\begin{enumerate}
\item J. K. Author, ``Title of thesis,'' M.S. thesis, Abbrev. Dept., Abbrev. Univ., City of Univ., Abbrev. State, year.
\item J. K. Author, ``Title of dissertation,'' Ph.D. dissertation, Abbrev. Dept., Abbrev. Univ., City of Univ., Abbrev. State, year.
\end{enumerate}
참조 \cite{b25,b26}.

\item {\bfseries 가장 일반적인 미출판 참고문헌 유형의 기본 형식:}
\begin{enumerate}
\item J. K. Author, private communication, Abbrev. Month, year.
\item J. K. Author, ``Title of paper,'' unpublished.
\item J. K. Author, ``Title of paper,'' to be published.
\end{enumerate}
참조 \cite{b27}--\cite{b29}.

\item {\bfseries 표준의 기본 형식:}
\begin{enumerate}
\item \emph{Title of Standard}, Standard number, date.
\item \emph{Title of Standard}, Standard number, Corporate author, location, date.
\end{enumerate}
참조 \cite{b30,b31}.

\item {\bfseries 참고문헌 예시의 기사 번호:}\\
참조 \cite{b32,b33}.

\item {\bfseries et al. 사용 예시:}\\
참조 \cite{b34}.

\end{itemize}


\section*{Acknowledgment}
미국 영어에서 ``acknowledgment''의 선호되는 철자는 ``g'' 뒤에 ``e''가 없는 것입니다. 감사의 말이 많더라도 단수 제목을 사용하십시오. ``One of us (S.B.A.) would like to thank $\ldots$ .''와 같은 표현은 피하십시오. 대신 ``F. A. Author thanks $\ldots$ .''라고 쓰십시오. 대부분의 경우 스폰서 및 재정 지원 감사는 여기가 아니라 첫 페이지의 번호 없는 각주에 배치됩니다.


\begin{thebibliography}{00}

\bibitem{b1} G. O. Young, ``Synthetic structure of industrial plastics,'' in \emph{Plastics,} 2\textsuperscript{nd} ed., vol. 3, J. Peters, Ed. New York, NY, USA: McGraw-Hill, 1964, pp. 15--64.

\bibitem{b2} W.-K. Chen, \emph{Linear Networks and Systems.} Belmont, CA, USA: Wadsworth, 1993, pp. 123--135.

\bibitem{b3} J. U. Duncombe, ``Infrared navigation---Part I: An assessment of feasibility,'' \emph{IEEE Trans. Electron Devices}, vol. ED-11, no. 1, pp. 34--39, Jan. 1959, 10.1109/TED.2016.2628402.

\bibitem{b4} E. P. Wigner, ``Theory of traveling-wave optical laser,'' \emph{Phys. Rev}., vol. 134, pp. A635--A646, Dec. 1965.

\bibitem{b5} E. H. Miller, ``A note on reflector arrays,'' \emph{IEEE Trans. Antennas Propagat}., to be published.

\bibitem{b6} E. E. Reber, R. L. Michell, and C. J. Carter, ``Oxygen absorption in the earth's atmosphere,'' Aerospace Corp., Los Angeles, CA, USA, Tech. Rep. TR-0200 (4230-46)-3, Nov. 1988.

\bibitem{b7} J. H. Davis and J. R. Cogdell, ``Calibration program for the 16-foot antenna,'' Elect. Eng. Res. Lab., Univ. Texas, Austin, TX, USA, Tech. Memo. NGL-006-69-3, Nov. 15, 1987.

\bibitem{b8} \emph{Transmission Systems for Communications}, 3\textsuperscript{rd} ed., Western Electric Co., Winston-Salem, NC, USA, 1985, pp. 44--60.

\bibitem{b9} \emph{Motorola Semiconductor Data Manual}, Motorola Semiconductor Products Inc., Phoenix, AZ, USA, 1989.

\bibitem{b10} G. O. Young, ``Synthetic structure of industrial
plastics,'' in Plastics, vol. 3, Polymers of Hexadromicon, J. Peters,
Ed., 2\textsuperscript{nd} ed. New York, NY, USA: McGraw-Hill, 1964, pp. 15-64.
[Online]. Available:
\url{http://www.bookref.com}.

\bibitem{b11} \emph{The Founders' Constitution}, Philip B. Kurland
and Ralph Lerner, eds., Chicago, IL, USA: Univ. Chicago Press, 1987.
[Online]. Available: \url{http://press-pubs.uchicago.edu/founders/}

\bibitem{b12} The Terahertz Wave eBook. ZOmega Terahertz Corp., 2014.
[Online]. Available:
\url{http://dl.z-thz.com/eBook/zomegaebookpdf_1206_sr.pdf}. Accessed on: May 19, 2014.

\bibitem{b13} Philip B. Kurland and Ralph Lerner, eds., \emph{The
Founders' Constitution.} Chicago, IL, USA: Univ. of Chicago Press,
1987, Accessed on: Feb. 28, 2010, [Online] Available:
\url{http://press-pubs.uchicago.edu/founders/}

\bibitem{b14} J. S. Turner, ``New directions in communications,'' \emph{IEEE J. Sel. Areas Commun}., vol. 13, no. 1, pp. 11-23, Jan. 1995.

\bibitem{b15} W. P. Risk, G. S. Kino, and H. J. Shaw, ``Fiber-optic frequency shifter using a surface acoustic wave incident at an oblique angle,'' \emph{Opt. Lett.}, vol. 11, no. 2, pp. 115--117, Feb. 1986.

\bibitem{b16} P. Kopyt \emph{et al., ``}Electric properties of graphene-based conductive layers from DC up to terahertz range,'' \emph{IEEE THz Sci. Technol.,} to be published. DOI: 10.1109/TTHZ.2016.2544142.

\bibitem{b17} PROCESS Corporation, Boston, MA, USA. Intranets:
Internet technologies deployed behind the firewall for corporate
productivity. Presented at INET96 Annual Meeting. [Online].
Available: \url{http://home.process.com/Intranets/wp2.htp}

\bibitem{b18} R. J. Hijmans and J. van Etten, ``Raster: Geographic analysis and modeling with raster data,'' R Package Version 2.0-12, Jan. 12, 2012. [Online]. Available: \url{http://CRAN.R-project.org/package=raster}

\bibitem{b19} Teralyzer. Lytera UG, Kirchhain, Germany [Online].
Available:
\url{http://www.lytera.de/Terahertz_THz_Spectroscopy.php?id=home}, Accessed on: Jun. 5, 2014.

\bibitem{b20} U.S. House. 102\textsuperscript{nd} Congress, 1\textsuperscript{st} Session. (1991, Jan. 11). \emph{H. Con. Res. 1, Sense of the Congress on Approval of}  \emph{Military Action}. [Online]. Available: LEXIS Library: GENFED File: BILLS

\bibitem{b21} Musical toothbrush with mirror, by L.M.R. Brooks. (1992, May 19). Patent D 326 189 [Online]. Available: NEXIS Library: LEXPAT File: DES

\bibitem{b22} D. B. Payne and J. R. Stern, ``Wavelength-switched pas- sively coupled single-mode optical network,'' in \emph{Proc. IOOC-ECOC,} Boston, MA, USA, 1985, pp. 585--590.

\bibitem{b23} D. Ebehard and E. Voges, ``Digital single sideband detection for interferometric sensors,'' presented at the \emph{2\textsuperscript{nd} Int. Conf. Optical Fiber Sensors,} Stuttgart, Germany, Jan. 2-5, 1984.

\bibitem{b24} G. Brandli and M. Dick, ``Alternating current fed power supply,'' U.S. Patent 4 084 217, Nov. 4, 1978.

\bibitem{b25} J. O. Williams, ``Narrow-band analyzer,'' Ph.D. dissertation, Dept. Elect. Eng., Harvard Univ., Cambridge, MA, USA, 1993.

\bibitem{b26} N. Kawasaki, ``Parametric study of thermal and chemical nonequilibrium nozzle flow,'' M.S. thesis, Dept. Electron. Eng., Osaka Univ., Osaka, Japan, 1993.

\bibitem{b27} A. Harrison, private communication, May 1995.

\bibitem{b28} B. Smith, ``An approach to graphs of linear forms,'' unpublished.

\bibitem{b29} A. Brahms, ``Representation error for real numbers in binary computer arithmetic,'' IEEE Computer Group Repository, Paper R-67-85.

\bibitem{b30} IEEE Criteria for Class IE Electric Systems, IEEE Standard 308, 1969.

\bibitem{b31} Letter Symbols for Quantities, ANSI Standard Y10.5-1968.

\bibitem{b32} R. Fardel, M. Nagel, F. Nuesch, T. Lippert, and A. Wokaun, ``Fabrication of organic light emitting diode pixels by laser-assisted forward transfer,'' \emph{Appl. Phys. Lett.}, vol. 91, no. 6, Aug. 2007, Art. no. 061103.~

\bibitem{b33} J. Zhang and N. Tansu, ``Optical gain and laser characteristics of InGaN quantum wells on ternary InGaN substrates,'' \emph{IEEE Photon. J.}, vol. 5, no. 2, Apr. 2013, Art. no. 2600111

\bibitem{b34} S. Azodolmolky~\emph{et al.}, Experimental demonstration of an impairment aware network planning and operation tool for transparent/translucent optical networks,''~\emph{J. Lightw. Technol.}, vol. 29, no. 4, pp. 439--448, Sep. 2011.

\end{thebibliography}

\begin{IEEEbiography}[{\includegraphics[width=1in,height=1.25in,clip,keepaspectratio]{author1.png}}]{First A. Author}는 2001년 샬롯츠빌에 있는 버지니아 대학교에서 항공우주 공학 학사 및 석사 학위를 받았으며, 2008년 펜실베이니아주 필라델피아에 있는 드렉셀 대학교에서 기계 공학 박사 학위를 받았습니다.
2001년부터 2004년까지 프린스턴 플라즈마 물리학 연구소에서 연구 조교로 근무했습니다. 2009년부터 텍사스 A{\&}M 대학교 기계 공학과 조교수로 재직 중입니다. 그는 3권의 책, 150편 이상의 논문, 70건 이상의 발명품을 저술했습니다. 그의 연구 관심 분야는 고압 및 고밀도 비열 플라즈마 방전 공정 및 응용, 마이크로스케일 플라즈마 방전, 액체 내 방전, 분광 진단, 플라즈마 추진 및 혁신 플라즈마 응용입니다. 그는 저널 \emph{Earth, Moon, Planets}의 부편집장이며 두 개의 특허를 보유하고 있습니다.

Author 박사는 2008년 국제 지자기 및 항공학 협회 우수 젊은 과학자상과 2011년 IEEE 전자기 호환성 학회 최우수 심포지엄 논문상을 수상했습니다.
\end{IEEEbiography}


\begin{IEEEbiography}[{\includegraphics[width=1in,height=1.25in,clip,keepaspectratio]{author2.png}}]{Second B. Author} (M'76--SM'81--F'87) 및 모든 저자는 약력을 포함할 수 있습니다. 약력은 컨퍼런스 관련 논문에는 포함되지 않는 경우가 많습니다. 이 저자는 1976년에 IEEE 회원(M), 1981년에 시니어 회원(SM), 1987년에 펠로우(F)가 되었습니다. 첫 번째 단락에는 출생지 및/또는 날짜(장소, 날짜 순)가 포함될 수 있습니다. 다음으로 저자의 학력이 나열됩니다. 학위는 학위 유형, 전공 분야, 기관, 도시, 주, 국가 및 학위 취득 연도와 함께 나열되어야 합니다. 저자의 전공 분야는 소문자로 표기해야 합니다.

두 번째 단락은 저자의 성이 아닌 인칭 대명사(그 또는 그녀)를 사용합니다. 여기에는 여름 및 펠로우십 직업을 포함한 군 복무 및 직장 경력이 나열됩니다. 직함은 대문자로 표기합니다. 현재 직업에는 위치가 있어야 합니다. 이전 직책은 위치 없이 나열될 수 있습니다. 이전 출판물에 관한 정보가 포함될 수 있습니다. 3권 이상의 책이나 출판된 기사를 나열하지 않도록 하십시오. 약력 내에 책의 출판사를 나열하는 형식은 참고문헌과 유사하게 책 제목(출판사 이름, 연도)입니다. 현재 및 이전 연구 관심사가 단락을 끝맺습니다.

세 번째 단락은 저자의 직함과 성(예: Dr. Smith, Prof. Jones, Mr. Kajor, Ms. Hunter)으로 시작합니다. IEEE 이외의 전문 학회 멤버십을 나열하십시오. 마지막으로 IEEE 위원회 및 출판물에 대한 수상 및 활동을 나열하십시오. 사진이 제공되는 경우 품질이 좋고 전문적으로 보여야 합니다. 다음은 저자 약력의 두 가지 예입니다.
\end{IEEEbiography}

\newpage

%사진이 없거나 포함하고 싶지 않은 경우 아래와 같이 IEEEbiographynophoto를 사용할 수 있습니다.

\begin{IEEEbiographynophoto}{Third C. Author, Jr.} (M'87)는 2004년 대만 자이의 국립 중정 대학교에서 기계 공학 학사 학위를, 2006년 대만 신주의 국립 칭화 대학교에서 기계 공학 석사 학위를 받았습니다. 그는 현재 미국 텍사스주 칼리지 스테이션에 있는 텍사스 A{\&}M 대학교에서 기계 공학 박사 과정을 밟고 있습니다.

2008년부터 2009년까지 대만 타이베이의 중앙연구원 물리학 연구소에서 연구 조교로 근무했습니다. 그의 연구 관심 분야는 비열 대기압 플라즈마를 이용한 표면 처리 및 생물학적/의학적 치료 기술 개발, 플라즈마 소스에 대한 기초 연구, 마이크로 또는 나노 구조 표면 제작 등입니다.

Author 씨의 수상 경력으로는 Frew Fellowship(호주 과학 아카데미), I. I. Rabi Prize(APS), European Frequency and Time Forum Award, Carl Zeiss Research Award, William F. Meggers Award 및 Adolph Lomb Medal(OSA)이 있습니다.
\end{IEEEbiographynophoto}


\EOD

\end{document}
