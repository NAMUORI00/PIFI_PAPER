\begin{IEEEbiography}[{\includegraphics[width=1in,height=1.25in,clip,keepaspectratio]{author1.png}}]{First A. Author}는 2001년 샬롯츠빌에 있는 버지니아 대학교에서 항공우주 공학 학사 및 석사 학위를 받았으며, 2008년 펜실베이니아주 필라델피아에 있는 드렉셀 대학교에서 기계 공학 박사 학위를 받았습니다.
2001년부터 2004년까지 프린스턴 플라즈마 물리학 연구소에서 연구 조교로 근무했습니다. 2009년부터 텍사스 A{\&}M 대학교 기계 공학과 조교수로 재직 중입니다. 그는 3권의 책, 150편 이상의 논문, 70건 이상의 발명품을 저술했습니다. 그의 연구 관심 분야는 고압 및 고밀도 비열 플라즈마 방전 공정 및 응용, 마이크로스케일 플라즈마 방전, 액체 내 방전, 분광 진단, 플라즈마 추진 및 혁신 플라즈마 응용입니다. 그는 저널 \emph{Earth, Moon, Planets}의 부편집장이며 두 개의 특허를 보유하고 있습니다.

Author 박사는 2008년 국제 지자기 및 항공학 협회 우수 젊은 과학자상과 2011년 IEEE 전자기 호환성 학회 최우수 심포지엄 논문상을 수상했습니다.
\end{IEEEbiography}


\begin{IEEEbiography}[{\includegraphics[width=1in,height=1.25in,clip,keepaspectratio]{author2.png}}]{Second B. Author} (M'76--SM'81--F'87) 및 모든 저자는 약력을 포함할 수 있습니다. 약력은 컨퍼런스 관련 논문에는 포함되지 않는 경우가 많습니다. 이 저자는 1976년에 IEEE 회원(M), 1981년에 시니어 회원(SM), 1987년에 펠로우(F)가 되었습니다. 첫 번째 단락에는 출생지 및/또는 날짜(장소, 날짜 순)가 포함될 수 있습니다. 다음으로 저자의 학력이 나열됩니다. 학위는 학위 유형, 전공 분야, 기관, 도시, 주, 국가 및 학위 취득 연도와 함께 나열되어야 합니다. 저자의 전공 분야는 소문자로 표기해야 합니다.

두 번째 단락은 저자의 성이 아닌 인칭 대명사(그 또는 그녀)를 사용합니다. 여기에는 여름 및 펠로우십 직업을 포함한 군 복무 및 직장 경력이 나열됩니다. 직함은 대문자로 표기합니다. 현재 직업에는 위치가 있어야 합니다. 이전 직책은 위치 없이 나열될 수 있습니다. 이전 출판물에 관한 정보가 포함될 수 있습니다. 3권 이상의 책이나 출판된 기사를 나열하지 않도록 하십시오. 약력 내에 책의 출판사를 나열하는 형식은 참고문헌과 유사하게 책 제목(출판사 이름, 연도)입니다. 현재 및 이전 연구 관심사가 단락을 끝맺습니다.

세 번째 단락은 저자의 직함과 성(예: Dr. Smith, Prof. Jones, Mr. Kajor, Ms. Hunter)으로 시작합니다. IEEE 이외의 전문 학회 멤버십을 나열하십시오. 마지막으로 IEEE 위원회 및 출판물에 대한 수상 및 활동을 나열하십시오. 사진이 제공되는 경우 품질이 좋고 전문적으로 보여야 합니다. 다음은 저자 약력의 두 가지 예입니다.
\end{IEEEbiography}

\newpage

%사진이 없거나 포함하고 싶지 않은 경우 아래와 같이 IEEEbiographynophoto를 사용할 수 있습니다.

\begin{IEEEbiographynophoto}{Third C. Author, Jr.} (M'87)는 2004년 대만 자이의 국립 중정 대학교에서 기계 공학 학사 학위를, 2006년 대만 신주의 국립 칭화 대학교에서 기계 공학 석사 학위를 받았습니다. 그는 현재 미국 텍사스주 칼리지 스테이션에 있는 텍사스 A{\&}M 대학교에서 기계 공학 박사 과정을 밟고 있습니다.

2008년부터 2009년까지 대만 타이베이의 중앙연구원 물리학 연구소에서 연구 조교로 근무했습니다. 그의 연구 관심 분야는 비열 대기압 플라즈마를 이용한 표면 처리 및 생물학적/의학적 치료 기술 개발, 플라즈마 소스에 대한 기초 연구, 마이크로 또는 나노 구조 표면 제작 등입니다.

Author 씨의 수상 경력으로는 Frew Fellowship(호주 과학 아카데미), I. I. Rabi Prize(APS), European Frequency and Time Forum Award, Carl Zeiss Research Award, William F. Meggers Award 및 Adolph Lomb Medal(OSA)이 있습니다.
\end{IEEEbiographynophoto}
