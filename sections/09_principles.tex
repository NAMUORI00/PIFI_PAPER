\section{\break Publication Principles}
저자는 다음 사항을 고려해야 합니다.
\begin{enumerate}
\item 출판을 위해 제출된 기술 논문은 지식의 상태를 발전시켜야 하며 관련 선행 연구를 인용해야 합니다.
\item 제출된 논문의 길이는 연구의 중요성에 비례하거나 복잡성에 적절해야 합니다. 예를 들어, 이전에 출판된 연구의 명백한 확장은 출판에 적합하지 않거나 몇 페이지만으로 충분히 다룰 수 있을지 모릅니다.
\item 저자는 동료 심사 위원과 편집자 모두에게 논문의 과학적 및 기술적 가치를 설득해야 합니다. 놀랍거나 예상치 못한 결과가 보고될 때는 입증 기준이 더 높습니다.
\item 과학적 진보를 위해서는 재현이 필요하므로, 출판을 위해 제출된 논문은 독자가 유사한 실험이나 계산을 수행하고 보고된 결과를 사용할 수 있도록 충분한 정보를 제공해야 합니다. 모든 것을 공개할 필요는 없지만, 논문에는 새롭고 유용하며 완전히 설명된 정보가 포함되어야 합니다. 예를 들어, 논문의 주된 목적이 새로운 측정 기술을 소개하는 것이라면 시료의 화학적 조성을 보고할 필요는 없습니다. 결과가 적절한 데이터와 중요한 세부 사항으로 뒷받침되지 않는 경우 저자는 심사 위원으로부터 이의 제기를 받을 것으로 예상해야 합니다.
\item 진행 중인 작업을 설명하거나 최신 기술 성과를 발표하는 논문은 전문 컨퍼런스 발표에는 적합할 수 있지만 출판에는 적합하지 않을 수 있습니다.
\end{enumerate}
