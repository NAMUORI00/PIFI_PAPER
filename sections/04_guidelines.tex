\section{Guidelines for Graphics Preparation and Submission}
\label{sec:guidelines}

\subsection{Types of Graphics}
다음 목록은 IEEE 저널에 게재되는 다양한 유형의 그래픽을 요약한 것입니다. 구성 및 색상/회색 음영 사용에 따라 분류됩니다.

\subsubsection{Color/Grayscale figures}
{컬러 또는 검정/회색 음영으로 나타나도록 의도된 그림입니다. 이러한 그림에는 사진, 일러스트레이션, 다색 그래프 및 순서도가 포함될 수 있습니다. 다색 그래프의 경우 회색 배경이나 음영, 스크린샷은 피하고 대신 데이터를 수집하는 데 사용된 프로그램에서 그래프를 내보내십시오.}

\subsubsection{Line Art figures}
{검은색 선과 도형으로만 구성된 그림입니다. 이러한 그림에는 회색 음영이나 하프톤이 없어야 하며 검은색과 흰색만 있어야 합니다.}

\subsubsection{Author photos}
{저자 사진은 참고문헌 아래 기사 끝에 위치한 저자 약력과 함께 포함되어야 합니다.}

\subsubsection{Tables}
{일반적으로 흑백이지만 때로는 색상이 포함된 데이터 차트입니다.}

\begin{table}
\caption{\textbf{자기 특성 단위}}
\label{table}
\setlength{\tabcolsep}{3pt}
\begin{tabular}{|p{25pt}|p{75pt}|p{115pt}|}
\hline
기호&
수량&
가우스 및 \par CGS EMU에서 SI로의 변환 $^{\mathrm{a}}$ \\
\hline
$\Phi $&
자속&
1 Mx $\to  10^{-8}$ Wb $= 10^{-8}$ V$\cdot $s \\
$B$&
자속 밀도, \par 자기 유도&
1 G $\to  10^{-4}$ T $= 10^{-4}$ Wb/m$^{2}$ \\
$H$&
자기장 강도&
1 Oe $\to  10^{3}/(4\pi )$ A/m \\
$m$&
자기 모멘트&
1 erg/G $=$ 1 emu \par $\to 10^{-3}$ A$\cdot $m$^{2} = 10^{-3}$ J/T \\
$M$&
자화&
1 erg/(G$\cdot $cm$^{3}) =$ 1 emu/cm$^{3}$ \par $\to 10^{3}$ A/m \\
4$\pi M$&
자화&
1 G $\to  10^{3}/(4\pi )$ A/m \\
$\sigma $&
비자화&
1 erg/(G$\cdot $g) $=$ 1 emu/g $\to $ 1 A$\cdot $m$^{2}$/kg \\
$j$&
자기 쌍극자 \par 모멘트&
1 erg/G $=$ 1 emu \par $\to 4\pi \times  10^{-10}$ Wb$\cdot $m \\
$J$&
자기 분극&
1 erg/(G$\cdot $cm$^{3}) =$ 1 emu/cm$^{3}$ \par $\to 4\pi \times  10^{-4}$ T \\
$\chi , \kappa $&
자화율&
1 $\to  4\pi $ \\
$\chi_{\rho }$&
질량 자화율&
1 cm$^{3}$/g $\to  4\pi \times  10^{-3}$ m$^{3}$/kg \\
$\mu $&
투자율&
1 $\to  4\pi \times  10^{-7}$ H/m \par $= 4\pi \times  10^{-7}$ Wb/(A$\cdot $m) \\
$\mu_{r}$&
상대 투자율&
322: $\mu \to \mu_{r}$ \\
$w, W$&
에너지 밀도&
1 erg/cm$^{3} \to  10^{-1}$ J/m$^{3}$ \\
$N, D$&
반자화 계수&
1 $\to  1/(4\pi )$ \\
\hline
\multicolumn{3}{p{251pt}}{표의 세로선은 선택 사항입니다. 전체 표의 캡션 역할을 하는 문장에는 각주 문자가 필요하지 않습니다.}\\
\multicolumn{3}{p{251pt}}{$^{\mathrm{a}}$가우스 단위는 정자기에 대한 cgs emu와 동일합니다. Mx
$=$ 맥스웰, G $=$ 가우스, Oe $=$ 에르스테드; Wb $=$ 웨버, V $=$ 볼트, s $=$
초, T $=$ 테슬라, m $=$ 미터, A $=$ 암페어, J $=$ 줄, kg $=$
킬로그램, H $=$ 헨리.}
\end{tabular}
\label{tab1}
\end{table}

\subsection{Multipart figures}
둘 이상의 하위 그림이 나란히 또는 쌓여서 구성된 그림입니다. 멀티파트 그림이 여러 그림 유형(한 부분은 라인 아트이고 다른 부분은 회색 음영 또는 컬러)으로 구성된 경우 그림은 더 엄격한 지침을 충족해야 합니다.

\subsection{File Formats For Graphics}\label{formats}
적절한 그래픽 처리 프로그램을 사용하여 그래픽의 형식을 지정하고 저장하십시오. 이 프로그램을 사용하면 이미지를 PostScript(.PS), Encapsulated PostScript(.EPS), Tagged Image File Format(.TIFF), Portable Document Format(.PDF), Portable Network Graphics(.PNG) 또는 Metapost(.MPS)로 생성하고 크기를 조정하고 해상도 설정을 조정할 수 있습니다. 최종 논문을 제출할 때 모든 그래픽은 원고와 함께 이러한 형식 중 하나로 개별적으로 제출해야 합니다.

\subsection{Sizing of Graphics}
대부분의 차트, 그래프 및 표는 1열 너비(3.5인치/88밀리미터/21파이카) 또는 페이지 너비(7.16인치/181밀리미터/43파이카)입니다. 그래픽의 최대 깊이는 8.5인치(216밀리미터/54파이카)일 수 있습니다. 그래픽의 깊이를 선택할 때는 캡션을 위한 공간을 확보하십시오. 저자가 선택하는 경우 그림 크기를 열 너비와 페이지 너비 사이로 조정할 수 있지만, 필요한 경우가 아니면 그림 크기를 열 너비보다 작게 조정하지 않는 것이 좋습니다.

현재 위에 나열된 것과 일치하지 않는 열 치수를 가진 간행물이 하나 있습니다. Proceedings of the IEEE의 열 치수는 3.25인치(82.5밀리미터/19.5파이카)입니다.

저자 사진의 최종 인쇄 크기는 정확히 가로 1인치, 세로 1.25인치(25.4밀리미터$\,\times\,$31.75밀리미터/6파이카$\,\times\,$7.5파이카)입니다. 사설에 인쇄된 저자 사진은 가로 1.59인치, 세로 2인치(40밀리미터$\,\times\,$50밀리미터/9.5파이카$\,\times\,$12파이카)입니다.

\subsection{Resolution }
그림의 적절한 해상도는 ``그림 유형'' 섹션에 정의된 대로 그림 유형에 따라 달라집니다. 저자 사진, 컬러 및 회색 음영 그림은 최소 300dpi여야 합니다. 표를 포함한 라인 아트는 최소 600dpi여야 합니다.

\subsection{Vector Art}
여러 컴퓨터 플랫폼에서 그림의 무결성을 보존하기 위해 .EPS/.PDF/.PS 형식의 파일을 허용합니다. 최상의 품질 결과를 얻으려면 모든 글꼴을 포함하거나 텍스트를 윤곽선으로 변환해야 합니다.

\subsection{Color Space}
색 공간이라는 용어는 해당 매체 내에서 표현할 수 있는 색상의 전체 합계를 나타냅니다. 우리의 목적을 위해 세 가지 주요 색 공간은 회색 음영, RGB(빨강/초록/파랑) 및 CMYK(시안/마젠타/노랑/검정)입니다. RGB는 일반적으로 화면 그래픽에 사용되는 반면 CMYK는 인쇄 목적으로 사용됩니다.

모든 컬러 그림은 RGB 또는 CMYK 색 공간에서 생성되어야 합니다. 회색 음영 이미지는 회색 음영 색 공간으로 제출해야 합니다. 라인 아트는 회색 음영 또는 비트맵 색 공간으로 제공될 수 있습니다. ``비트맵 색 공간''과 ``비트맵 파일 형식''은 같은 것이 아닙니다. 비트맵 색 공간을 선택한 경우 .TIF/.TIFF/.PNG가 권장되는 파일 형식입니다.

\subsection{Accepted Fonts Within Figures}
그래픽을 준비할 때 IEEE는 다음 오픈 타입 글꼴 중 하나를 사용할 것을 제안합니다: Times New Roman, Helvetica, Arial, Cambria 및 Symbol. EPS, PS 또는 PDF 파일을 제공하는 경우 모든 글꼴이 포함되어야 합니다. 일부 글꼴은 운영 체제에만 고유할 수 있습니다. 글꼴이 포함되지 않으면 그래픽의 일부가 왜곡되거나 누락될 수 있습니다.

그림을 완성할 때 안전한 옵션은 파일을 저장하기 전에 글꼴을 제거하여 ``윤곽선'' 유형을 만드는 것입니다. 이렇게 하면 글꼴이 아트워크로 변환되어 모든 화면에서 균일하게 나타납니다.

\subsection{Using Labels Within Figures}

\subsubsection{Figure Axis labels }
그림 축 라벨은 종종 혼란의 원인이 됩니다. 기호보다는 단어를 사용하십시오. 예를 들어, 단순히 ``M''이 아니라 ``Magnetization'' 또는 ``Magnetization M''이라고 쓰십시오. 단위는 괄호 안에 넣으십시오. 축에 단위만 라벨로 붙이지 마십시오. 예를 들어 그림 1과 같이 단순히 ``A/m''이 아니라 ``Magnetization (A/m)'' 또는 ``Magnetization (A$\cdot$m$^{-1}$)''이라고 쓰십시오. 축에 수량과 단위의 비율로 라벨을 붙이지 마십시오. 예를 들어 ``Temperature/K''가 아니라 ``Temperature (K)''라고 쓰십시오.

승수는 특히 혼란스러울 수 있습니다. ``Magnetization (kA/m)'' 또는 ``Magnetization (10$^{3}$ A/m)''라고 쓰십시오. ``Magnetization (A/m)$\,\times\,$1000''이라고 쓰지 마십시오. 독자가 그림 1의 상단 축 라벨이 16000 A/m를 의미하는지 0.016 A/m를 의미하는지 알 수 없기 때문입니다. 그림 라벨은 읽기 쉬워야 하며 약 8~10포인트 크기여야 합니다.

\subsubsection{Subfigure Labels in Multipart Figures and Tables}
멀티파트 그림은 최종 제출 전에 결합하고 라벨을 붙여야 합니다. 라벨은 (a) (b) (c) 형식의 8포인트 Times New Roman 글꼴로 각 하위 그림 아래 중앙에 표시되어야 합니다.

\subsection{File Naming}
그림(라인 아트워크 또는 사진)은 저자의 성의 처음 5글자로 시작하여 이름을 지정해야 합니다. 파일 이름의 다음 문자는 기사에서 이 이미지의 순차적 위치를 나타내는 숫자여야 합니다. 예를 들어, 저자 ``Anderson''의 논문에서 처음 세 그림의 이름은 ander1.tif, ander2.tif, ander3.ps가 됩니다.

표는 표의 본문(캡션 아님)만 포함해야 하며, 저자의 이름과 표 번호 사이에 `.t`가 삽입된다는 점을 제외하고는 그림과 유사하게 이름을 지정해야 합니다. 예를 들어, 저자 Anderson의 처음 세 표의 이름은 ander.t1.tif, ander.t2.ps, ander.t3.eps가 됩니다.

저자 사진은 사진에 찍힌 저자의 성의 처음 5글자를 사용하여 이름을 지정해야 합니다. 예를 들어, 논문의 저자 사진 4장은 oppen.ps, moshc.tif, chen.eps, duran.pdf로 명명될 수 있습니다.

두 명 이상의 저자가 같은 성을 가진 경우, 구별이 될 때까지 성의 다섯 번째, 네 번째, 세 번째$\ldots$ 문자 대신 첫 번째 이니셜을 대체할 수 있습니다. 예를 들어, 두 저자 Michael과 Monica Oppenheimer의 사진은 oppmi.tif와 oppmo.eps로 명명됩니다.

\subsection{Referencing a Figure or Table Within Your Paper}
논문 내에서 그림과 표를 참조할 때는 문장의 시작 부분이라도 ``Fig.''라는 약어를 사용하십시오. 그림은 아라비아 숫자로 번호를 매겨야 합니다. ``Table''을 줄여 쓰지 마십시오. 표는 로마 숫자로 번호를 매겨야 합니다.

\subsection{Submitting Your Graphics}
그림은 섹션 IV-C에 나열된 파일 형식 중 하나로 원고와 별도로 개별적으로 제출해야 합니다. 그림 캡션은 그림 아래에 배치하고 표 제목은 표 위에 배치하십시오. 캡션을 그림의 일부로 포함하거나 그림에 연결된 ``텍스트 상자''에 넣지 마십시오. 또한 그림 외부에 테두리를 두지 마십시오.

\subsection{Color Processing/Printing in IEEE Journals}
모든 IEEE Transactions, Journals 및 Letters는 저자가 IEEE {\it Xplore}$\circledR$에 컬러 그림을 무료로 게시할 수 있도록 허용하며, 인쇄 버전의 경우 자동으로 회색 음영으로 변환합니다. 대부분의 저널에서 저자가 선택하는 경우 그림과 표를 컬러로 인쇄할 수도 있습니다. 이 서비스에는 저자에게 추가 비용이 발생한다는 점에 유의하십시오. 컬러 그래픽 인쇄를 원하는 경우, 최종 논문에 어떤 그림이나 표를 그렇게 처리할지 명시하고 추가 비용을 지불할 의사가 있음을 명시한 메모를 포함하십시오.
