\raggedbottom
\section{Reference Examples}

\begin{itemize}

\item {\bfseries 책의 기본 형식:}\\
J. K. Author, ``Title of chapter in the book,'' in \emph{Title of His Published Book, x}th ed. City of Publisher, (only U.S. State), Country: Abbrev. of Publisher, year, ch. $x$, sec. $x$, pp. \emph{xxx--xxx.}\\
참조 \cite{b1,b2}.

\item {\bfseries 정기간행물의 기본 형식:}\\
J. K. Author, ``Name of paper,'' \emph{Abbrev. Title of Periodical}, vol. \emph{x, no}. $x, $pp\emph{. xxx--xxx, }Abbrev. Month, year, DOI. 10.1109.\emph{XXX}.123456.\\
참조 \cite{b3}--\cite{b5}.

\item {\bfseries 보고서의 기본 형식:}\\
J. K. Author, ``Title of report,'' Abbrev. Name of Co., City of Co., Abbrev. State, Country, Rep. \emph{xxx}, year.\\
참조 \cite{b6,b7}.

\item {\bfseries 핸드북의 기본 형식:}\\
\emph{Name of Manual/Handbook, x} ed., Abbrev. Name of Co., City of Co., Abbrev. State, Country, year, pp. \emph{xxx--xxx.}\\
참조 \cite{b8,b9}.

\item {\bfseries 책의 기본 형식(온라인 이용 가능 시):}\\
J. K. Author, ``Title of chapter in the book,'' in \emph{Title of
Published Book}, $x$th ed. City of Publisher, State, Country: Abbrev.
of Publisher, year, ch. $x$, sec. $x$, pp. \emph{xxx--xxx}. [Online].
Available: \url{http://www.web.com}\\
참조 \cite{b10}--\cite{b13}.

\item {\bfseries 저널의 기본 형식(온라인 이용 가능 시):}\\
J. K. Author, ``Name of paper,'' \emph{Abbrev. Title of Periodical}, vol. $x$, no. $x$, pp. \emph{xxx--xxx}, Abbrev. Month, year. Accessed on: Month, Day, year, DOI: 10.1109.\emph{XXX}.123456, [Online].\\
참조 \cite{b14}--\cite{b16}.

\item {\bfseries 컨퍼런스 발표 논문의 기본 형식(온라인 이용 가능 시): }\\
J.K. Author. (year, month). Title. presented at abbrev. conference title. [Type of Medium]. Available: site/path/file\\
참조 \cite{b17}.

\item {\bfseries 보고서 및 핸드북의 기본 형식(온라인 이용 가능 시):}\\
J. K. Author. ``Title of report,'' Company. City, State, Country. Rep. no., (optional: vol./issue), Date. [Online] Available: site/path/file\\
참조 \cite{b18,b19}.

\item {\bfseries 컴퓨터 프로그램 및 전자 문서의 기본 형식(온라인 이용 가능 시): }\\
Legislative body. Number of Congress, Session. (year, month day). \emph{Number of bill or resolution}, \emph{Title}. [Type of medium]. Available: site/path/file\\
참조 \cite{b20}.

\item {\bfseries 특허의 기본 형식(온라인 이용 가능 시):}\\
Name of the invention, by inventor's name. (year, month day). Patent Number [Type of medium]. Available: site/path/file\\
참조 \cite{b21}.

\item {\bfseries 컨퍼런스 프로시딩의 기본 형식(출판됨):}\\
J. K. Author, ``Title of paper,'' in \emph{Abbreviated Name of Conf.}, City of Conf., Abbrev. State (if given), Country, year, pp. \emph{xxxxxx.}\\
참조 \cite{b22}.

\item {\bfseries 컨퍼런스 발표 논문의 예(미출판):}\\
참조 \cite{b23}.

\item {\bfseries 특허의 기본 형식}$:$\\
J. K. Author, ``Title of patent,'' U.S. Patent \emph{x xxx xxx}, Abbrev. Month, day, year.\\
참조 \cite{b24}.

\item {\bfseries 학위 논문(석사) 및 박사 학위 논문의 기본 형식:}
\begin{enumerate}
\item J. K. Author, ``Title of thesis,'' M.S. thesis, Abbrev. Dept., Abbrev. Univ., City of Univ., Abbrev. State, year.
\item J. K. Author, ``Title of dissertation,'' Ph.D. dissertation, Abbrev. Dept., Abbrev. Univ., City of Univ., Abbrev. State, year.
\end{enumerate}
참조 \cite{b25,b26}.

\item {\bfseries 가장 일반적인 미출판 참고문헌 유형의 기본 형식:}
\begin{enumerate}
\item J. K. Author, private communication, Abbrev. Month, year.
\item J. K. Author, ``Title of paper,'' unpublished.
\item J. K. Author, ``Title of paper,'' to be published.
\end{enumerate}
참조 \cite{b27}--\cite{b29}.

\item {\bfseries 표준의 기본 형식:}
\begin{enumerate}
\item \emph{Title of Standard}, Standard number, date.
\item \emph{Title of Standard}, Standard number, Corporate author, location, date.
\end{enumerate}
참조 \cite{b30,b31}.

\item {\bfseries 참고문헌 예시의 기사 번호:}\\
참조 \cite{b32,b33}.

\item {\bfseries et al. 사용 예시:}\\
참조 \cite{b34}.

\end{itemize}


\section*{Acknowledgment}
미국 영어에서 ``acknowledgment''의 선호되는 철자는 ``g'' 뒤에 ``e''가 없는 것입니다. 감사의 말이 많더라도 단수 제목을 사용하십시오. ``One of us (S.B.A.) would like to thank $\ldots$ .''와 같은 표현은 피하십시오. 대신 ``F. A. Author thanks $\ldots$ .''라고 쓰십시오. 대부분의 경우 스폰서 및 재정 지원 감사는 여기가 아니라 첫 페이지의 번호 없는 각주에 배치됩니다.


\begin{thebibliography}{00}

\bibitem{b1} G. O. Young, ``Synthetic structure of industrial plastics,'' in \emph{Plastics,} 2\textsuperscript{nd} ed., vol. 3, J. Peters, Ed. New York, NY, USA: McGraw-Hill, 1964, pp. 15--64.

\bibitem{b2} W.-K. Chen, \emph{Linear Networks and Systems.} Belmont, CA, USA: Wadsworth, 1993, pp. 123--135.

\bibitem{b3} J. U. Duncombe, ``Infrared navigation---Part I: An assessment of feasibility,'' \emph{IEEE Trans. Electron Devices}, vol. ED-11, no. 1, pp. 34--39, Jan. 1959, 10.1109/TED.2016.2628402.

\bibitem{b4} E. P. Wigner, ``Theory of traveling-wave optical laser,'' \emph{Phys. Rev}., vol. 134, pp. A635--A646, Dec. 1965.

\bibitem{b5} E. H. Miller, ``A note on reflector arrays,'' \emph{IEEE Trans. Antennas Propagat}., to be published.

\bibitem{b6} E. E. Reber, R. L. Michell, and C. J. Carter, ``Oxygen absorption in the earth's atmosphere,'' Aerospace Corp., Los Angeles, CA, USA, Tech. Rep. TR-0200 (4230-46)-3, Nov. 1988.

\bibitem{b7} J. H. Davis and J. R. Cogdell, ``Calibration program for the 16-foot antenna,'' Elect. Eng. Res. Lab., Univ. Texas, Austin, TX, USA, Tech. Memo. NGL-006-69-3, Nov. 15, 1987.

\bibitem{b8} \emph{Transmission Systems for Communications}, 3\textsuperscript{rd} ed., Western Electric Co., Winston-Salem, NC, USA, 1985, pp. 44--60.

\bibitem{b9} \emph{Motorola Semiconductor Data Manual}, Motorola Semiconductor Products Inc., Phoenix, AZ, USA, 1989.

\bibitem{b10} G. O. Young, ``Synthetic structure of industrial
plastics,'' in Plastics, vol. 3, Polymers of Hexadromicon, J. Peters,
Ed., 2\textsuperscript{nd} ed. New York, NY, USA: McGraw-Hill, 1964, pp. 15-64.
[Online]. Available:
\url{http://www.bookref.com}.

\bibitem{b11} \emph{The Founders' Constitution}, Philip B. Kurland
and Ralph Lerner, eds., Chicago, IL, USA: Univ. Chicago Press, 1987.
[Online]. Available: \url{http://press-pubs.uchicago.edu/founders/}

\bibitem{b12} The Terahertz Wave eBook. ZOmega Terahertz Corp., 2014.
[Online]. Available:
\url{http://dl.z-thz.com/eBook/zomegaebookpdf_1206_sr.pdf}. Accessed on: May 19, 2014.

\bibitem{b13} Philip B. Kurland and Ralph Lerner, eds., \emph{The
Founders' Constitution.} Chicago, IL, USA: Univ. of Chicago Press,
1987, Accessed on: Feb. 28, 2010, [Online] Available:
\url{http://press-pubs.uchicago.edu/founders/}

\bibitem{b14} J. S. Turner, ``New directions in communications,'' \emph{IEEE J. Sel. Areas Commun}., vol. 13, no. 1, pp. 11-23, Jan. 1995.

\bibitem{b15} W. P. Risk, G. S. Kino, and H. J. Shaw, ``Fiber-optic frequency shifter using a surface acoustic wave incident at an oblique angle,'' \emph{Opt. Lett.}, vol. 11, no. 2, pp. 115--117, Feb. 1986.

\bibitem{b16} P. Kopyt \emph{et al., ``}Electric properties of graphene-based conductive layers from DC up to terahertz range,'' \emph{IEEE THz Sci. Technol.,} to be published. DOI: 10.1109/TTHZ.2016.2544142.

\bibitem{b17} PROCESS Corporation, Boston, MA, USA. Intranets:
Internet technologies deployed behind the firewall for corporate
productivity. Presented at INET96 Annual Meeting. [Online].
Available: \url{http://home.process.com/Intranets/wp2.htp}

\bibitem{b18} R. J. Hijmans and J. van Etten, ``Raster: Geographic analysis and modeling with raster data,'' R Package Version 2.0-12, Jan. 12, 2012. [Online]. Available: \url{http://CRAN.R-project.org/package=raster}

\bibitem{b19} Teralyzer. Lytera UG, Kirchhain, Germany [Online].
Available:
\url{http://www.lytera.de/Terahertz_THz_Spectroscopy.php?id=home}, Accessed on: Jun. 5, 2014.

\bibitem{b20} U.S. House. 102\textsuperscript{nd} Congress, 1\textsuperscript{st} Session. (1991, Jan. 11). \emph{H. Con. Res. 1, Sense of the Congress on Approval of}  \emph{Military Action}. [Online]. Available: LEXIS Library: GENFED File: BILLS

\bibitem{b21} Musical toothbrush with mirror, by L.M.R. Brooks. (1992, May 19). Patent D 326 189 [Online]. Available: NEXIS Library: LEXPAT File: DES

\bibitem{b22} D. B. Payne and J. R. Stern, ``Wavelength-switched pas- sively coupled single-mode optical network,'' in \emph{Proc. IOOC-ECOC,} Boston, MA, USA, 1985, pp. 585--590.

\bibitem{b23} D. Ebehard and E. Voges, ``Digital single sideband detection for interferometric sensors,'' presented at the \emph{2\textsuperscript{nd} Int. Conf. Optical Fiber Sensors,} Stuttgart, Germany, Jan. 2-5, 1984.

\bibitem{b24} G. Brandli and M. Dick, ``Alternating current fed power supply,'' U.S. Patent 4 084 217, Nov. 4, 1978.

\bibitem{b25} J. O. Williams, ``Narrow-band analyzer,'' Ph.D. dissertation, Dept. Elect. Eng., Harvard Univ., Cambridge, MA, USA, 1993.

\bibitem{b26} N. Kawasaki, ``Parametric study of thermal and chemical nonequilibrium nozzle flow,'' M.S. thesis, Dept. Electron. Eng., Osaka Univ., Osaka, Japan, 1993.

\bibitem{b27} A. Harrison, private communication, May 1995.

\bibitem{b28} B. Smith, ``An approach to graphs of linear forms,'' unpublished.

\bibitem{b29} A. Brahms, ``Representation error for real numbers in binary computer arithmetic,'' IEEE Computer Group Repository, Paper R-67-85.

\bibitem{b30} IEEE Criteria for Class IE Electric Systems, IEEE Standard 308, 1969.

\bibitem{b31} Letter Symbols for Quantities, ANSI Standard Y10.5-1968.

\bibitem{b32} R. Fardel, M. Nagel, F. Nuesch, T. Lippert, and A. Wokaun, ``Fabrication of organic light emitting diode pixels by laser-assisted forward transfer,'' \emph{Appl. Phys. Lett.}, vol. 91, no. 6, Aug. 2007, Art. no. 061103.~

\bibitem{b33} J. Zhang and N. Tansu, ``Optical gain and laser characteristics of InGaN quantum wells on ternary InGaN substrates,'' \emph{IEEE Photon. J.}, vol. 5, no. 2, Apr. 2013, Art. no. 2600111

\bibitem{b34} S. Azodolmolky~\emph{et al.}, Experimental demonstration of an impairment aware network planning and operation tool for transparent/translucent optical networks,''~\emph{J. Lightw. Technol.}, vol. 29, no. 4, pp. 439--448, Sep. 2011.

\end{thebibliography}
